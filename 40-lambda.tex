% vim: spell spelllang=en_us

\RequirePackage[l2tabu,orthodox]{nag}
\RequirePackage{fixltx2e}

\documentclass[a4paper,12pt]{article}

\usepackage{fourier}
%\usepackage{amsmath}
\usepackage{empheq}

\title{\textbf{Lambdas Everywhere}}
\author{Peter H.\ Fr{\"o}hlich\\
phf@acm.org}

\begin{document}

\maketitle

\begin{abstract}
Tries to show how you can \(\lambda\) it up.
Sorry, this is nowhere near complete,
but I don't have time for more right now.
\end{abstract}

\section*{Problem 1}

You can state short fragments like \(S \equiv \lambda nab.a(nab)\) in the text
itself.
Or you can typeset them more prominently as follows:
\[
S \equiv \lambda nab.a(nab)
\]
If need be, you can even number your equations:
\begin{equation}
S \equiv \lambda nab.a(nab)
\label{Slambda}
\end{equation}
Then you can later refer to equation (\ref{Slambda}) and impress your friends
with your \LaTeX{} power.
Got a \emph{really} important result? Frame it!
\begin{empheq}[box=\fbox]{equation}
S \equiv \lambda nab.a(nab)
\end{empheq}
That's about it for just plain \(\lambda\) expressions.

\section*{Problem 2}

One of \emph{many} ways to typeset a derivation.
Note the use of ``\(\equiv\)'' for steps that just rename things versus the use
of ``\(\to\)'' for reductions.
%
\begin{align*}
S3
\equiv {} & (\lambda nab.a(nab)) 3 \\
\equiv {} & (\lambda nab.a(nab)) (\lambda sz.s(s(s(z)))) \\
\to {}    & \lambda ab.a((\lambda sz.s(s(s(z))))ab) \\
\to {}    & \lambda ab.a((\lambda z.a(a(a(z))))b) \\
\to {}    & \lambda ab.a(a(a(a(b)))) \\
\equiv {} & \lambda sz.s(s(s(s(z)))) \\
\equiv {} & 4
\end{align*}
%
If you want to be even clearer, underline the redexes for reductions and use a
fancy font for abbreviations:
%
\begin{align*}
\textsf{S3}
\equiv {} & (\lambda nab.a(nab)) \textsf{3} \\
\equiv {} & (\lambda \underline{n}ab.a(nab)) \underline{(\lambda sz.s(s(s(z))))} \\
\to {}    & \lambda ab.a((\lambda \underline{s}z.s(s(s(z))))\underline{a}b) \\
\to {}    & \lambda ab.a((\lambda \underline{z}.a(a(a(z))))\underline{b}) \\
\to {}    & \lambda ab.a(a(a(a(b)))) \\
\equiv {} & \lambda sz.s(s(s(s(z)))) \\
\equiv {} & \textsf{4}
\end{align*}
%
And we're done for now.

\end{document}
