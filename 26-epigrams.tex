% A little something I felt I had to do. :-)

\RequirePackage[l2tabu,orthodox]{nag}
\RequirePackage{fixltx2e}

\documentclass[letterpaper,11pt]{article}

\usepackage[margin=0.53in,includefoot]{geometry}
\usepackage{fourier}
\usepackage{multicol}
\usepackage{microtype}
\usepackage{enumitem}

\title{\textbf{Epigrams on Programming}}
\author{Alan J.\ Perlis\\(April 1, 1922--February 7, 1990)}
\date{September 1982}

\begin{document}
\thispagestyle{empty}

\maketitle

\begin{multicols}{2}
\noindent
The phenomena surrounding computers are diverse and yield a
surprisingly rich base for launching metaphors at individual
and group activities.
%
Conversely, classical human endeavors provide an inexhaustible
source of metaphor for those of us who are in labor within
computation.
%
Such relationships between society and device are not new, but
the incredible growth of the computer's influence (both real
and implied) lends this symbiotic dependency a vitality like
a gangly youth growing out of his clothes within an endless puberty.

The epigrams that follow attempt to capture some of the
dimensions of this traffic in imagery that sharpens,
focuses, clarifies, enlarges and beclouds our view of
this most remarkable of all mans' artifacts, the computer.

\begin{enumerate}[wide=0pt,label=\textbf{\arabic*.}]
\item One man's constant is another man's variable.
\item Functions delay binding: data structures induce binding.
Moral: Structure data late in the programming process.
\item Syntactic sugar causes cancer of the semi-colons.
\item Every program is a part of some other program and rarely fits.
\item If a program manipulates a large amount of data,
it does so in a small number of ways.
\item Symmetry is a complexity reducing concept (co-routines include sub-routines); seek it everywhere.
\item It is easier to write an incorrect program than understand a correct one.
\item A programming language is low level when its programs require attention to the irrelevant.
\item It is better to have 100 functions operate on one data structure than 10 functions on 10 data structures.
\item Get into a rut early: Do the same processes the same way. Accumulate idioms. Standardize. The only difference (!) between Shakespeare and you was the size of his idiom list --- not the size of his vocabulary.
\item If you have a procedure with 10 parameters, you probably missed some.
\item Recursion is the root of computation since it trades description for time.
\item If two people write exactly the same program, each should be put in micro-code and then they certainly won't be the same.
\item In the long run every program becomes rococo --- then rubble.
\item Everything should be built top-down, except the first time.
\item Every program has (at least) two purposes: the one for which it was written and another for which it wasn't.
\item If a listener nods his head when you're explaining your program, wake him up.
\item A program without a loop and a structured variable isn't worth writing.
\item A language that doesn't affect the way you think about programming, is not worth knowing.
\item Wherever there is modularity there is the potential for misunderstanding: Hiding information implies a need to check communication.
\item Optimization hinders evolution.
\item A good system can't have a weak command language.
\item To understand a program you must become both the machine and the program.
\item Perhaps if we wrote programs from childhood on, as adults we'd be able to read them.
\item One can only display complex information in the mind. Like seeing, movement or flow or alteration of view is more important than the static picture, no matter how lovely.
\item There will always be things we wish to say in our programs that in all known languages can only be said poorly.
\item Once you understand how to write a program get someone else to write it.
\item Around computers it is difficult to find the correct unit of time to measure progress. Some cathedrals took a century to complete. Can you imagine the grandeur and scope of a program that would take as long?
\item For systems, the analogue of a face-lift is to add to the control graph an edge that creates a cycle, not just an additional node.
\item In programming, everything we do is a special case of something more general --- and often we know it too quickly.
\item Simplicity does not precede complexity, but follows it.
\item Programmers are not to be measured by their ingenuity and their logic but by the completeness of their case analysis.
\item The 11th commandment was ``Thou Shalt Compute'' or ``Thou Shalt Not Compute'' --- I forget which.
\item The string is a stark data structure and everywhere it is passed there is much duplication of process. It is a perfect vehicle for hiding information.
\item Everyone can be taught to sculpt: Michelangelo would have had to be taught how not to. So it is with the great programmers.
\item The use of a program to prove the 4-color theorem will not change mathematics --- it merely demonstrates that the theorem, a challenge for a century, is probably not important to mathematics.
\item The most important computer is the one that rages in our skulls and ever seeks that satisfactory external emulator. The standardization of real computers would be a disaster --- and so it probably won't happen.
\item Structured Programming supports the law of the excluded muddle.
\item Re graphics: A picture is worth 10K words --- but only those to describe the picture. Hardly any sets of 10K words can be adequately described with pictures.
\item There are two ways to write error-free programs; only the third one works.
\item Some programming languages manage to absorb change, but withstand progress.
\item You can measure a programmer's perspective by noting his attitude on the continuing vitality of FORTRAN.
\item In software systems it is often the early bird that makes the worm.
\item Sometimes I think the only universal in the computing field is the fetch-execute-cycle.
\item The goal of computation is the emulation of our synthetic abilities, not the understanding of our analytic ones.
\item Like punning, programming is a play on words.
\item As Will Rogers would have said, ``There is no such thing as a free variable.''
\item The best book on programming for the layman is ``Alice in Wonderland''; but that's because it's the best book on anything for the layman.
\item Giving up on assembly language was the apple in our Garden of Eden: Languages whose use squanders machine cycles are sinful. The LISP machine now permits LISP programmers to abandon bra and fig-leaf.
\item When we understand knowledge-based systems, it will be as before --- except our finger-tips will have been singed.
\item Bringing computers into the home won't change either one, but may revitalize the corner saloon.
\item Systems have sub-systems and sub-systems have sub-systems and so on ad infinitum --- which is why we're always starting over.
\item So many good ideas are never heard from again once they embark in a voyage on the semantic gulf.
\item Beware of the Turing tar-pit in which everything is possible but nothing of interest is easy.
\item A LISP programmer knows the value of everything, but the cost of nothing.
\item Software is under a constant tension. Being symbolic it is arbitrarily perfectible; but also it is arbitrarily changeable.
\item It is easier to change the specification to fit the program than vice versa.
\item Fools ignore complexity. Pragmatists suffer it. Some can avoid it. Geniuses remove it.
\item In English every word can be verbed. Would that it were so in our programming languages.
\item Dana Scott is the Church of the Lattice-Way Saints.
\item In programming, as in everything else, to be in error is to be reborn.
\item In computing, invariants are ephemeral.
\item When we write programs that ``learn,'' it turns out we do and they don't.
\item Often it is means that justify ends: Goals advance technique and technique survives even when goal structures crumble.
\item Make no mistake about it: Computers process numbers --- not symbols. We measure our understanding (and control) by the extent to which we can arithmetize an activity.
\item Making something variable is easy. Controlling duration of constancy is the trick.
\item Think of all the psychic energy expended in seeking a fundamental distinction between ``algorithm'' and ``program.''
\item If we believe in data structures, we must believe in independent (hence simultaneous) processing. For why else would we collect items within a structure? Why do we tolerate languages that give us the one without the other?
\item In a 5 year period we get one superb programming language. Only we can't control when the 5 year period will begin.
\item Over the centuries the Indians developed sign language for communicating phenomena of interest. Programmers from different tribes (FORTRAN, LISP, ALGOL, SNOBOL, etc.) could use one that doesn't require them to carry a blackboard on their ponies.
\item Documentation is like term insurance: It satisfies because almost no one who subscribes to it depends on its benefits.
\item An adequate bootstrap is a contradiction in terms.
\item It is not a language's weaknesses but its strengths that control the gradient of its change: Alas, a language never escapes its embryonic sac.
\item It is possible that software is not like anything else, that it is meant to be discarded: that the whole point is to always see it as soap bubble?
\item Because of its vitality, the computing field is always in desperate need of new cliches: Banality soothes our nerves.
\item It is the user who should parameterize procedures, not their creators.
\item The cybernetic exchange between man, computer and algorithm is like a game of musical chairs: The frantic search for balance always leaves one of the three standing ill at ease.
\item If your computer speaks English it was probably made in Japan.
\item A year spent in artificial intelligence is enough to make one believe in God.
\item Prolonged contact with the computer turns mathematicians into clerks and vice versa.
\item In computing, turning the obvious into the useful is a living definition of the word ``frustration.''
\item We are on the verge: Today our program proved Fermat's next-to-last theorem!
\item What is the difference between a Turing machine and the modern computer? It's the same as that between Hillary's ascent of Everest and the establishment of a Hilton hotel on its peak.
\item Motto for a research laboratory: What we work on today, others will first think of tomorrow.
\item Though the Chinese should adore APL, it's FORTRAN they put their money on.
\item We kid ourselves if we think that the ratio of procedure to data in an active data-base system can be made arbitrarily small or even kept small.
\item We have the mini and the micro computer. In what semantic niche would the pico computer fall?
\item It is not the computer's fault that Maxwell's equations are not adequate to design the electric motor.
\item One does not learn computing by using a hand calculator, but one can forget arithmetic.
\item Computation has made the tree flower.
\item The computer reminds one of Lon Chaney --- it is the machine of a thousand faces.
\item The computer is the ultimate polluter. Its feces are indistinguishable from the food it produces.
\item When someone says ``I want a programming language in which I need only say what I wish done,'' give him a lollipop.
\item Interfaces keep things tidy, but don't accelerate growth: Functions do.
\item Don't have good ideas if you aren't willing to be responsible for them.
\item Computers don't introduce order anywhere as much as they expose opportunities.
\item When a professor insists computer science is X but not Y, have compassion for his graduate students.
\item In computing, the mean time to failure keeps getting shorter.
\item In man-machine symbiosis, it is man who must adjust: The machines can't.
\item We will never run out of things to program as long as there is a single program around.
\item Dealing with failure is easy: Work hard to improve. Success is also easy to handle: You've solved the wrong problem. Work hard to improve.
\item One can't proceed from the informal to the formal by formal means.
\item Purely applicative languages are poorly applicable.
\item The proof of a system's value is its existence.
\item You can't communicate complexity, only an awareness of it.
\item It's difficult to extract sense from strings, but they're the only communication coin we can count on.
\item The debate rages on: Is PL/I Bactrian or Dromedary?
\item Whenever two programmers meet to criticize their programs, both are silent.
\item Think of it! With VLSI we can pack 100 ENIACs in 1 sq.\ cm.
\item Editing is a rewording activity.
\item Why did the Roman Empire collapse? What is the Latin for office automation?
\item Computer Science is embarrassed by the computer.
\item The only constructive theory connecting neuroscience and psychology will arise from the study of software.
\item Within a computer natural language is unnatural.
\item Most people find the concept of programming obvious, but the doing impossible.
\item You think you know when you learn, are more sure when you can write, even more when you can teach, but certain when you can program.
\item It goes against the grain of modern education to teach children to program. What fun is there in making plans, acquiring discipline in organizing thoughts, devoting attention to detail and learning to be self-critical?
\item If you can imagine a society in which the computer-robot is the only menial, you can imagine anything.
\item Programming is an unnatural act.
\item Adapting old programs to fit new machines usually means adapting new machines to behave like old ones.
\item In seeking the unattainable, simplicity only gets in the way.
\end{enumerate}
%
If there are epigrams, there must be meta-epigrams.
%
\begin{enumerate}[resume,wide=0pt,label=\textbf{\arabic*.}]
\item Epigrams are interfaces across which appreciation and insight flow.
\item Epigrams parameterize auras.
\item Epigrams are macros, since they are executed at read time.
\item Epigrams crystallize incongruities.
\item Epigrams retrieve deep semantics from a data base that is all procedure.
\item Epigrams scorn detail and make a point: They are a superb high-level documentation.
\item Epigrams are more like vitamins than protein.
\item Epigrams have extremely low entropy.
\item The last epigram? Neither eat nor drink them, snuff epigrams.
\end{enumerate}
\end{multicols}

\end{document}
