% vim: spell spelllang=en_us

% enable useful warnings about various things
\RequirePackage[l2tabu,orthodox]{nag}
% fixes for various LaTeX2e kernel bugs
\RequirePackage{fixltx2e}

% explicit paper size (same default), larger font (default 10pt)
\documentclass[letterpaper,12pt]{article}

% replace the main font (default Computer Modern)
\usepackage{fourier}

% lots and lots of additional math-related commands
\usepackage{amsmath}

% make the title bold
\title{\textbf{A Fancier Article on Some Topic}}
% explicit line-break, typewriter font for email address,
% more authors
\author{Peter H.\ Fr{\"o}hlich\\
\texttt{phf@cs.jhu.edu}
\and
Someone E.\ Entirely\\
\texttt{someone@example.net}
\and
Can Never Recall\\
\texttt{recall@what.org}}

\begin{document}

\maketitle

\begin{abstract}
Oh the wonders of \LaTeX!
As you can see a lot of the layout is completely
automatic and moderately boring.
But that's sort of the point.
\end{abstract}

\section{Introduction}

Where you introduce the topic you want to rant
about for the next few pages.
And of course you can cite things like
% citations by keyword, see below, need to run
% twice to resolve properly
\cite{lamport94} or \cite{kant02} if you feel
like it.

\section{Methodology}

Where you describe the approach you took and
% footnote, note no space between word and command
attempt to justify\footnote{Check this out,
you can slap a footnote basically anywhere
and it'll actually come out alright. Well,
sorta anyway.} why it's valid.

\section{Results}

Where you give the results of your experiments
and explain away any anomalies that might be
embarrassing.

\section{Related Work}

Where you describe work that others have done
and either praise it or rip it to shreds.

\section{Conclusions}

Where you tell people that don't read the rest
what it all means; presumably also where you
state that more research is definitely needed.

\appendix

\section{Appendix}

Where you say whatever you couldn't say earlier
in the paper for some reason.
% here's some math since you've heard that TeX
% is good at that kind of stuff
There are silly formulas like \(a^2 + b^2 = c^2\) I
guess, but that's not nearly cool enough. So let's
try this instead:
\[
\sum_{n=1}^\infty \frac{(-1)^{n+1}}{n}
= 1 - \frac{1}{2} + \frac{1}{3} - \frac{1}{4} + \frac{1}{5} - \cdots
= \ln 2
\]
Not too shabby! And because I am sure that
\emph{someone} is taking Linear Algebra right
now:
\[
A_{m,n} =
\begin{pmatrix}
a_{1,1} & a_{1,2} & \cdots & a_{1,n}\\
a_{2,1} & a_{2,2} & \cdots & a_{2,n}\\
\vdots  & \vdots  & \ddots & \vdots\\
a_{m,1} & a_{m,2} & \cdots & a_{m,n}
\end{pmatrix}
\]
Doing this in regular \LaTeX{} would be a bit harder,
but with \AmS-\LaTeX{} it's no problem whatsoever.

% define entries of the bibliography, this is the
% silly way of doing bibliographies though...
\begin{thebibliography}{9}

\bibitem{kant02}
Immanuel Kant,
\emph{Prolegomena to Any Future Metaphysics
That Can Qualify as a Science}.
Translated by Paul Carus.
Open Court Publishing Company, Illinois,
1902.

\bibitem{lamport94}
Leslie Lamport,
\emph{\LaTeX: A Document Preparation System}.
Addison-Wesley, Massachusetts,
2nd edition,
1994.

\end{thebibliography}

\end{document}
